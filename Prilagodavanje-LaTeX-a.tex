\documentclass{beamer}
\usetheme{Boadilla}
\usepackage[croatian]{babel}
\usepackage[utf8]{inputenc}
\usecolortheme{lily}
\usepackage{amssymb}

\title{Prilagođavanje LaTeX-a}
\author{Alen Livojević \\ Dominik - Matej Čondrić}
\institute{Tehnički fakultet Rijeka}
\date{\today}

\begin{document}

\begin{frame}
\titlepage
\end{frame}

\begin{frame}[t]{Sadržaj}
\tableofcontents
\end{frame}

\section{Naredbe}

\begin{frame}[t]{Naredbe}
\begin{itemize}
\item Posebne riječi koje određuju ponašanje LaTeX-a
\item Uglavnom im prethodi kosa crta "\textbackslash", te ostali parametri
\item Pružaju laku mogućnost formatiranja dokumenta 
\item Sastoje od jednostavnih riječi kojima prethodi poseban znak 
\item Uz korištenje već poznatih naredbi, LaTeX nam pruža mogućnost stvaranja svojih naredbi 
\end{itemize}
\end{frame}

\begin{frame}[t]{Naredbe}
\begin{itemize}
	\item Neke naredbe zahtjevaju više parametara kako bi ispravno radile, na primjer :
		\begin{itemize}
			\item Naredbi "\textbackslash textbf" potrebno je u vitičaste zagrade dodati sadržaj koji želimo podebljati :
			\\
			"\textbackslash textbf\{Seminar\}" će ispisati kao "\textbf{Seminar}" \\
			\item Naredbi "\textbackslash begin" i "\textbackslash end" moramo dodati sadržaj kako bi započela neko okruženje
			\\
			"\textbackslash begin\{itemize\}" i "\textbackslash end\{itemize\}" će započeti okuženje za liste 
		\end{itemize}
	\item Također možemo dodati dodatne parametre naredbama kako bi promjenili njihovo ponašanje
		\begin{itemize}
			\item Dodamo li parametar "\textbackslash S" naredbi "\textbackslash item" promijeniti ćemo izgled natuknice
			\begin{itemize}
				\item[\S] Natuknica će izgledati ovako
			\end{itemize}	
		\end{itemize}		
\end{itemize}	
\end{frame}

\subsection{Dodavanje novih naredbi}

\begin{frame}[t]{Dodavanje novih naredbi}
\begin{itemize}
	\item LaTeX nam pruža lako dodavanje svojih naredbi pomoću naredbe "\textbackslash newcommand\{ime-naredbe\}\{funkcija\}"
	\item Primjer :
	\begin{itemize}
		\newcommand{\Q}{\mathbb{Q}} 
		\item Pomoću naredbe "\textbf{\textbackslash newcommand\{\textbackslash Q\}\{\textbackslash mathbb\{Q\}\}}" Q koji označava racionalne brojeve biti će ispisan ovako : \( \Q \)
	\end{itemize}
	\item Iako mjesto na kojoj izrađujemo novu naredbu nije bitno, savjetuje se da se nova naredba piše u preambuli	 
	\item Kao i kod drugih naredbi i kod stvaranja novih komandi možemo koristiti dodatni parametar koji će odrediti broj parametara 
	\item Primjer :
		\begin{itemize}
			\item \textbf{\textbackslash newcommand\{\textbackslash bb\}{[}2{]}\{\textbackslash mathbb\{\#2\}\}}
		\end{itemize}	
\end{itemize}	
\end{frame}

\bibliographystyle{ieeetr}
\bibliography{literatura}

\end{document}
