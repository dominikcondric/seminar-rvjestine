\documentclass{beamer}
\usetheme{Boadilla}
\usepackage[croatian]{babel}
\usepackage[utf8]{inputenc}

\title{Prilagođavanje LaTeX-a}
\author{Alen Livojević \\ Dominik - Matej Čondrić}
\institute{Tehnički fakultet Rijeka}
\date{\today}

\begin{document}

\begin{frame}
\titlepage
\end{frame}

\begin{frame}{Sadržaj}
\tableofcontents
\end{frame}

\section{Naredbe}

\begin{frame}[t]{Naredbe}
\begin{itemize}
\item Posebne riječi koje određuju ponašanje LaTeX-a
\item Uglavnom im prethodi kosa crta "\textbackslash", te ostali parametri
\item Pružaju laku mogućnost formatiranja dokumenta 
\item Sastoje od jednostavnih riječi kojima prethodi poseban znak 
\item Uz korištenje već poznatih naredbi, LaTeX nam pruža mogućnost stvaranja svojih naredbi 
\end{itemize}
\end{frame}

\begin{frame}{Naredbe}
\begin{itemize}
	\item Neke naredbe zahtjevaju više parametara kako bi ispravno radile, na primjer :
		\begin{itemize}
			\item Naredbi "\textbackslash textbf" potrebno je u vitičaste zagrade dodati sadržaj koji želimo podebljati :
			\\
			"\textbackslash textbf\{Seminar\}" će ispisati kao \textbf{seminar} \\
			\item Naredbi "\textbackslash begin" i \textbackslash end" moramo dodati sadržaj kako bi započela neko okruženje
		\end{itemize}	
\end{itemize}	
\end{frame}

\bibliographystyle{ieeetr}
\bibliography{literatura}

\end{document}
