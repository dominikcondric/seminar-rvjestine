\documentclass{beamer}
\usetheme{Boadilla}
\usepackage[croatian]{babel}
\usepackage[utf8]{inputenc}
\usecolortheme{lily}
\usepackage{amssymb}
\newcommand{\plusbinomial}[3][2]{(#2 + #3) ^ #1}

\title{Prilagođavanje LaTeX-a}
\author{Alen Livojević \\ Dominik - Matej Čondrić}
\institute{Tehnički fakultet Rijeka}
\date{\today}

\begin{document}

\begin{frame}
\titlepage
\end{frame}

\begin{frame}[t]{Sadržaj}
\tableofcontents
\end{frame}

\section{Naredbe}

\begin{frame}[t]{Naredbe}
\begin{itemize}
\item Posebne riječi koje određuju ponašanje LaTeX-a
\item Uglavnom im prethodi kosa crta "\textbackslash", te ostali parametri
\item Pružaju laku mogućnost formatiranja dokumenta 
\item Sastoje od jednostavnih riječi kojima prethodi poseban znak 
\item Uz korištenje već poznatih naredbi, LaTeX nam pruža mogućnost stvaranja svojih naredbi 
\end{itemize}
\end{frame}

\begin{frame}[t]{Naredbe}
\begin{itemize}
	\item Neke naredbe zahtjevaju više parametara kako bi ispravno radile, na primjer :
		\begin{itemize}
			\item Naredbi "\textbackslash textbf" potrebno je u vitičaste zagrade dodati sadržaj koji želimo podebljati :
			\\
			"\textbackslash textbf\{Seminar\}" će ispisati kao "\textbf{Seminar}" \\
			\item Naredbi "\textbackslash begin" i "\textbackslash end" moramo dodati sadržaj kako bi započela neko okruženje
			\\
			"\textbackslash begin\{itemize\}" i "\textbackslash end\{itemize\}" će započeti okuženje za liste 
		\end{itemize}
	\item Također možemo dodati dodatne parametre naredbama kako bi promjenili njihovo ponašanje
		\begin{itemize}
			\item Dodamo li parametar "\textbackslash S" naredbi "\textbackslash item" promijeniti ćemo izgled natuknice
			\begin{itemize}
				\item[\S] Natuknica će izgledati ovako
			\end{itemize}	
		\end{itemize}		
\end{itemize}	
\end{frame}

\subsection{Dodavanje novih naredbi}

\begin{frame}[t]{Dodavanje novih naredbi}
\begin{itemize}
	\item LaTeX nam pruža lako dodavanje svojih naredbi pomoću naredbe "\textbackslash newcommand\{ime-naredbe\}\{funkcija\}"
	\item Primjer :
	\begin{itemize}
		\newcommand{\Q}{\mathbb{Q}} 
		\item Pomoću naredbe "\textbf{\textbackslash newcommand\{\textbackslash Q\}\{\textbackslash mathbb\{Q\}\}}" Q koji označava racionalne brojeve biti će ispisan ovako : \( \Q \)
	\end{itemize}
	\item Iako mjesto na kojoj izrađujemo novu naredbu nije bitno, savjetuje se da se nova naredba piše u preambuli	 
	\item Kao i kod drugih naredbi i kod stvaranja novih komandi možemo koristiti dodatni parametar koji će odrediti broj parametara 
	\item Primjer :
		\begin{itemize}
			\item \textbf{\textbackslash newcommand\{\textbackslash bb\}{[}2{]}\{\textbackslash mathbb\{\#2\}\}}
		\end{itemize}	
\end{itemize}	
\end{frame}

\subsection{Naredbe s neobaveznim parametrima}

\begin{frame}[t]{Naredbe s neobaveznim parametrima}
\begin{itemize}
	\item Da bi poboljšali funkcioniranje naredbe dodajemo dodatne parametre u uglate zagrade
		\item Naredbu \textbf{\textbackslash newcommand\{\textbackslash plusbinomial\}{[}3{]}{[}2{]}\{(\#2 + \#3) \^\ \#1\}} će ispisati na sljedeći način :
				\[ \plusbinomial[4]{x}{y} \]
		\begin{itemize}
			\item \textbackslash plusbinomial označava ime nove naredbe		
			\item {[}3{]} označava broj parametara koje će naredba uzeti
			\item {[}2{]} je unaprijed definiran broj za prvi parametar
			\item  \{(\#2 + \#3) \^\ \#1\} je funckija nove naredbe, u ovom slučaju kvadrat zbroja
		\end{itemize}
\end{itemize}	
\end{frame}

\subsection{Stvaranje novih naredbi preko postojećih}

\begin{frame}[t]{Stvaranje novih naredbi preko postojećih}
\begin{itemize}
	\item Ukoliko želimo uvesti novu naredbu čije ime već postoji, LaTeX će javiti prilikom stvaranja dokumenta
	\item Da bi uveli novu naredbu s istim imenom preko već definirane naredbe koristimo naredbu "\textbackslash renewcommand" 
	\item Primjer : \\
			\textbf{\textbackslash renewcommand\{\textbackslash S\}\{\textbackslash mathbb\{S\}\}} \\
			Pozivom na naredbu \textbackslash ( \textbackslash S \textbackslash ) koja bi inače bila prevođena kao znak za beskonačnost ($\infty$) postaje znak 
			$\renewcommand{\S}{\mathbb{S}}$ 
										\( \S \)
										
\end{itemize}	
\end{frame}

\section{Okruženja}

\begin{frame}[t]{Okruženja}
\begin{itemize}
	\item Okruženja su razgraničena početnom oznakom \textbackslash begin i završnom oznakom \textbackslash end
	\item Primjer je tablica


\begin{tabular}{ c c c } 
  cell1 & cell2 & cell3 \\ 
  cell4 & cell5 & cell6 \\ 
  cell7 & cell8 & cell9 \\ 
 \end{tabular}

\item odnosno zapis u kodu:

\textbackslash begin\{tabular\}\{ c c c \} 
  cell1 \& cell2 \& cell3 \\ 
  cell4 \& cell5 \& cell6 \\ 
  cell7 \& cell8 \& cell9 \\ 
 end\{tabular\}


\end{itemize}
\end{frame}

\subsection{Definiranje novog okruženja}

\begin{frame}{Definiranje novog okruženja}
\begin{itemize}

\item Kao i kod naredbi, možete definirati nova okruženja
\item Nova definicija okruženja postiže se oznakom \textbackslash newenvironment
\item Primjer:
\textbackslash newenvironment\{boxed\}\\
\{\textbackslash begin\{center\}\\
    \textbackslash begin\{tabular\}\{|p\{0.9\textbackslash textwidth\}|\}\\
    \textbackslash hline\\
    \}\\
    \{ \\
    \textbackslash hline\\
    \textbackslash end\{tabular\} \\
    \textbackslash end\{center\}\\
    \}\\
    \item Ovo okruženje će nacrtati okvir oko teksta
\end{itemize}
\end{frame}

\begin{frame}
\begin{itemize}
	\item Odmah nakon novog okruženja, između zagrada, morate napisati ime okoline
	\item Unutar prvog para zagrada je postavljeno što će vaše novo okruženje učiniti prije nego što tekst bude unutar, a zatim unutar drugog para zagrada odredite što će vaše novo okruženje učiniti nakon teksta
\item Rezultat:
\begin{boxed}
T
\end{boxed}

	\item Mogu se definirati i okruženja koja prihvaćaju parametre

\end{itemize}
\end{frame}


\subsection{Numerirana okruženja}

\begin{frame}{Numerirana okruženja}
\begin{itemize}
	\item Numerirana okruženja mogu se kreirati ručno ili izravno pomoću naredbe \textbackslash newtheorem\\

\textbackslash newcounter\{example\}[section]\\
\textbackslash newenvironment\{example\}[1][]\{\textbackslash refstepcounter\{example\}\textbackslash par\textbackslash medskip\\
   \textbackslash noindent \textbackslash textbf\{Example~\textbackslash theexample. \#1\} \textbackslash rmfamily\}\{\textbackslash medskip\}\\
 
 

\textbackslash usepackage\{amsmath\}\\
\textbackslash newtheorem\{SampleEnv\}\{Sample Environment\}[section]\\

\end{itemize}
\end{frame}

\begin{frame}{Primjer:}
\newcounter{example1}[section]
\newenvironment{example1}[1][]{\refstepcounter{example1}\par\medskip
   \noindent \textbf{Primjer~\theexample. } \rmfamily}{\medskip}
   \begin{example1}
Numerirani dio
\end{example1}

\end{frame}
\section{Zaključak}
\begin{frame}[t]{Zaključak}
\begin{itemize}
	\item Naredbe nam pružaju laku mogućnost uređivanja i prilagođavanja sardžaja
	\item Naredbe mogu biti bez ili sa obaveznim parametrima, uz moguće dodavanje dodatnih parametara
	\item Uz već postojeće naredbe imamo mogućnost dodavati svoje ili promijeniti funkciju već postojećih naredbi
	\item Okruženje se koristi s odgovarajućim parom početnih i završnih izraza (\textbackslash begin i \textbackslash end)
	\item Oboje počinju i završavaju naziv okruženja kao argument u vitičastim zagradama
	\item Početni izraz može imati dodatne obvezne i / ili neobavezne argumente

\end{itemize}
\end{frame}


\bibliographystyle{ieeetr}
\bibliography{literatura}

\end{document}
